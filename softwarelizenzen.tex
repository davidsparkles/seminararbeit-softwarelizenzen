% Konfigurationsdatei f\"ur die Pfaddefinitionen einlesen
\input{se-wa-pfade}
%
%
% Festlegung der Sprache:
\newcommand{\seWaSprache}{deutsch}
%\newcommand{\seWaSprache}{englisch}

%
% Einlesen der .sty-Dateien
%
\input{\seWaPathSty/se-wa-input-styles-v0971}

%
% Individuelle Konfiguration des Dokumentes
%
\input{\seWaPathText/wa-konfiguration}

%
% Definition von Abk\"urzungen, Symbolen und eventuell Glossareintr\"agen
%
\input{\seWaPathText/wa-abkuerzungen}

\seIstSeminararbeit{}

\newcommand{\version}{0.971}

%
% Diese Redefinition ist nur f\"ur den Anhang der
% Vorlage (Hinweise zur Installation und \"Ubersetzung)
% notwendig; f\"ur Ihre Seminar-/Projekt-/Bachelorarbeit spielt sie keine Rolle
%
\renewcommand{\seVorlage}{\jobname}

% Verwendung kleinerer Schriftgroessen fŸr die Ueberschriften; sinnvoll bei kurzen Texten.
%
%
\KOMAoption{headings}{small}

% Fuer einen Kapitelanfang wird kein zusaetzlicher vertikaler Abstand erzeugt
%
%
% \seNoChapterSkip[-12.25mm]{}

\begin{document}

% Erzeugung des Titelblatts
%
%
%
\seTitelblattSeminararbeit[
%hilfslinien=ja,
%dhbwlogoSkalierung=0.5,
%dhbwlogoDeltaX=2.4,
%dhbwlogoDeltaY=-10,
studiengang=\seWirtschaftsinformatik,
%studienrichtung=\seApplicationManagement,
%studienrichtung=\seSalesUndConsulting,
studienrichtung=\seSoftwareEngineering,
%studienrichtung=\seSoftwaremethodik,
thema=Rechtliche Probleme im Bereich Open Source Software und gebraucher Software,
verfasser=Matthias Fabinski und David Hentschel,
%verfasserin= Melanie Musterfrau,
matrikelnummer= 1171695 und 3941419,
kurs=WWI\,13\,SE\,B,
firma=UnitCon GmbH und SAP SE,
% Da im Text ein Komma enthalten ist, muss der Text eingeklammert werden
%studiengangsleiterin=,
studiengangsleiter=Prof. Dr.-Ing. J\"org Baumgart,
modul=IT- \& Geschäftsprozess-Management,
lehrveranstaltung=IT-Recht,
%dozentin=,
dozent=Dr. Sven Mehlhorn
]


% Erzeugung der englischen Kurzfassung (Abstract); Verfasser, Firma und Thema werden automatisch \"ubernommen
%
% Der optionale Parameter kann verwendet werden, um f\"ur das Thema der Arbeit eine
% andere Formatierung vorzunehmen; das sollte in der Regel nicht erforderlich sein;
% ausserdem besteht die Gefahr inkonsistenter Titel auf dem Titelblatt und in der
% Kurzfassung
%
%
% Achtung: Das Kommando erzeugt nur dann eine Ausgabe, wenn \seWaSprache den Wert englisch besitzt
%
%
\seAbstract{} % dieses Kommando sollte standardm\"assig verwendet werden

%\seAbstract[\LaTeX-Vorlage zur Anfertigung \seThemaWaArbeit{} (Version \version{})]



% Erzeugung der Kurzfassung; Verfasser, Firma und Thema werden automatisch \"ubernommen
%
% Der optionale Parameter kann verwendet werden, um f\"ur das Thema der Arbeit eine
% andere Formatierung vorzunehmen; das sollte in der Regel nicht erforderlich sein;
% ausserdem besteht die Gefahr inkonsistenter Titel auf dem Titelblatt und in der
% Kurzfassung
%
%\seKurzfassung{} % dieses Kommando sollte standardm\"assig verwendet werden

%\seKurzfassung[\LaTeX-Vorlage zur Anfertigung einer Seminararbeit (Version \version{})]


% Beispiel f\"ur ein Kapitel, dass vor dem Einleitungskapitel kommt, z. B. ein Vorwort oder eine Danksagung
%\seKapitelVorEinleitung{Vorwort}
%
%Muss jetzt wirklich nicht sein, aber wenn Sie unbedingt (z. B.) Ihrem Haustier f\"ur die Unterst\"utzung bei
%der Anfertigung der Projektarbeit danken wollen ...

% 2012-02-06 Inhaltsverzeichnis muss vor den weiteren Verzeichnisses kommen
%
%
% Ausgabe des Inhaltsverzeichnisses
%
%
\seInhaltsverzeichnis[%
einrueckung=ja,
gliederungsebenen=4
]

%
%
% Wenn die Verzeichnisse (ohne Seitenvorschub) nach dem Inhaltsverzeichnis
% kommen sollen, sind die beiden folgenden Kommandos zu verwenden
%
% Ein neues Kapitel beginnt nicht auf einer neuen Seite
%
%
%\seChaptersWithoutNewpage{}

% Erzeugung eines vertikalen Abstands nach diesem Kapitel
%
%\seChapterEndSkip{}


% Ausgabe der verschiedenen Verzeichnisse
% abk: Abk\"urzungsverzeichnis
% sym: Symbolverzeichnis
% abb: Abbildungsverzeichnis
% tab: Tabellenverzeichnis
% prg: Listingverzeichnis
% alg: Algorithmenverzeichnis
%
%
% Achtung: Abk\"urzungs- und Symbolverzeichnis werden nur ausgegeben, wenn mindest ein Symbol bzw.
%                mindestens eine Abk\"urzung in der Arbeit verwendet wurden
%
%
% gliederungsebene:
% -- section: die Verzeichnisse werden einem Kapitel "Verzeichnisse" untergliedert
% -- chapter: die Verzeichnisse sind jeweils eigene Kapitel
% imInhaltsverzeichnis: ja/nein -- Sollen die Verzeichnisse im Inhaltsverzeichnis enthalten sein?
\seVerzeichnisse[gliederungsebene=section,imInhaltsverzeichnis=ja]{abk}{}{}{}{}




\seNewAcronymEntry{eula}{EULA}{Enduser Licence Agreement}{Enduser Licence Agreements}
\seNewAcronymEntry{bgh}{BGH}{Bundesgerichthof}{}
\seNewAcronymEntry{eugh}{EuGH}{Europäischen Gerichtshof}{}
\seNewAcronymEntry{oem}{OEM-Software}{Original Equipment Manufacturer Software}{}



%=================================================================
% Erstes eigentliches Kapitel der Arbeit; typischerweise das Einleitungskapitel;
% hier muss wieder auf die Nummierung mit arabischen Seitenzahlen umgestellt werden
%
\chapter{Einleitung}\pagenumbering{arabic}
\seChaptersWithoutNewpage{}
Das geistige Eigentum ist ein Gut und nach dem deutschen Gesetz schützenswert. Das bedeutet, dass die Urheber beziehungsweise Erfinder geistigen Eigentums, je nach Art bestimmte Rechte
an dem Eigentum selbst haben und gegebenenfalls Dritte von der Verwendung dieses Eigentums ausschließen oder einschränken. Insbesondere, wenn solch ein immaterielles Gut
ein Wirtschaftsgut ist, spielt der Schutz des geistigen Eigentums eine wichtige Rolle.

So wird in der Arbeit die Frage beleuchtet, wie Schutzrechte an Software entstehen und wie Nutzungsrechte an Software in Form von Lizenzverträgen eingeräumt werden.
Des Weiteren erklärt diese Seminararbeit, wie aus juristischer Sicht der Weiterverkauf von Softwarelizenzen funktioniert sowie was die Unterschiede zwischen
verschiedenen Open-Source Lizenzen sind. Zu diesem Zweck werden zunächst Kriterien aufgestellt, mit Hilfe derer ein Vergleich erstellt wird.

Als Quellen werden unter Anderem das Urhebergesetzbuch, das Patentgesetzbuch und das Designgesetzbuch herangezogen, wobei
diese durch Gerichtsurteile des Bundesgerichtshof sowie des Europäischen Gerichtshofes ergänzt werden. Desweiteren sind die Internetseiten ausgewählter
Open-Source Lizenzen die Quelle für deren Vergleich.

\seChapterEndSkip{}

\seChaptersNewpage{}

% Mit markright kann eine verk\"urzte Version der \"Uberschrift f\"ur den Seitenkopf generiert werden
%
%
%\markright{Formaler Aufbau}

\chapter{Grundlagen zu Softwarelizenzen}
\section{Grundlagen zu Lizenzen}
Die Vergabe von Lizenzen ist ein Mittel zur Übertragung von Rechten an nichtmateriellen Gütern vom Eigentümer auf einen anderen.

Diese Rechte sind durch das Gesetz in unterschiedliche Arten aufgeteilt, von denen das Patentrecht, das Gebrauchsmusterrecht, das Designrecht und das Urheberrecht im folgenden kurz beschrieben wird.
Sowohl Gebrauchsmuster als auch Patente beziehen sich auf Erfindungen die gemäß §1 Abs. 1 PatG und §1 Abs.1 GebrMG neu und gewerblich nutzbar sind,
wobei Patente auf “Erfindungen auf allen Gebieten der Technik” und Gebrauchsmuster auf Erfindungen, die “auf einem erfinderischen Schritt beruhen”, vergeben werden.
Weiterführend ist das Design gemäß §1 des Designgesetzes (kurz DesingG) das Erscheinungbild “eines ganzen Erzeugnisses oder eines Teiles davon” und ist gemäß §2 DesignG geschützt,
wenn es neu ist und Eigenart hat und zusätzlich dazu auch wie Patente und Gebrauchsmuster eingetragen ist. Im Gegensatz dazu werden Urheberrechte nicht in ein Register eingetragen und
beziehen sich gemäß des Urhebergesetzes (kurz UrhG) auf Werke der Literatur, Wissenschaft und Kultur, wozu gemäß §2 Abs. 1 UrhG auch Computerprogramme zählen.

Somit kann der Erfinder oder Schöpfer eines schützenswerten immateriellen Gutes gemäß der oben genannten Gesetze über seine Erfindung oder Werk verfügen,
wenn es entsprechenderweise eingetragen oder verbreitet ist. Ist die Erfindung letztendlich geschützt so kann zunächst der Eigentümer des Schutzrechtes ausschließlich über dieses verfügen.
Kann der Erfinder seine Erfindung selbst gewerblich verwerten, dann wäre es für ihn der günstigste Fall. Jedoch ist es häufig so, dass der Erfinder nicht das nötige Wissen und Kapital besitzt,
um seine Erfindung selbst gewerblich verwerten zu können, sodass es sich in solch einem Fall eher für ihn lohnt, das Schutzrecht oder das Nutzungsrecht zu verkaufen.

Zu diesem Zweck müssen Verkäufer und Käufer einen Verwertungsvertrag aushandeln, in dem die Vertragsleistungen, die Gegenleistungen und die Abreden vereinbart werden.
Dabei beschriebt die Vertragsleistung die Übertragung der Verwertungsrechte, die Gegenleistung wird meist in Form einer Zahlung erbracht und die Abrede enthält Informationen unter Anderem
zu Abrechnungsfristen, die Vertragslaufzeit oder auch ein Kündigungsrecht zum Aufheben des Verwertungsvertrages. Bei der Erstellung eines solchen Verwertungsvertrages sollte darauf geachtet werden,
ihn möglichst klar, ausführlich und vor Allem unter Berücksichtigung aller erdenklichen Möglichkeiten zu formulieren, um so späteren Konflikten durch “Vertragslücken” vorzubeugen.
Wird in dem Verwertungsvertrag eine unbeschränkte Rechteübertragung vereinbart, so werden dem Vertragspartner nicht nur die Benutzungsrechte, sondern auch die Verfügungsrechte übertragen.
Das bedeutet, dass der Vertragspartner nach Erhalt der Verfügungsrechte, beispielsweise eines Patentes, dieses Patent weiterverkaufen könnte. Um weiterhin das Eigentum an den Verfügungsrechten
zu behalten, werden lediglich die Benutzungsrechte, auch Lizenzrechte genannt, in einem beschränkten Verwertungsvertrag übertragen.

Lizenzrechte können unter verschiedenen Beschränkungen übertragen werden, sodass Lizenzenrechte häufig zeitlich, örtlich und beispielsweise auch nach der Benutzunsgsart eingeschränkt werden kann.
Neben den eben genannten Einschränkungen werden Lizenzrechte auch in ausschließliche und einfache Lizenzen unterschieden, wobei bei ausschließlichen Lizenzen der Lizenznehmer als einziger meist
auch unter Ausschluss des Lizenzgebers die Benutzungsrechte hält. Im Gegensatz dazu kann der Lizenzgeber bei einer einfachen Lizenz diese weiterhin selbst nutzen und auch weiteren Lizenznehmern
anbieten. Beispielsweise überträgt ein Autor seinem Verlag im Regelfall ein ausschließliches Benutzungsrecht sein Buch zu drucken, um somit auszuschließen, dass das Werk auch von anderen Verlägen
gedruckt wird. Demgegenüber wird eine einfaches Benutzungsrecht von zum Beispiel einen Softwarehersteller auf seine Kunden übertragen, damit gewährleistet ist, dass der Softwarehersteller mehrere
Lizenzen vergeben kann.

Die freie Gestaltung von Lizenzverträgen wird durch Vorschriften gegen Wettbewerbsbeschränkung seitens des Gesetzes gegen Wettbewerbsbeschränkung in Deutschland und auch seitens des EG-Vertrages
eingeschränkt. In dem EG-Vertrag heißt es beispielsweise, dass “alle Vereinbarungen zwischen Unternehmen [...] mit dem Gemeinsamen Markt unvereinbar und verboten sind, [...] welche den Handel
zwischen Mitgliedstaaten zu beeinträchtigen geeignet sind”. Somit können Lizenzverträge nicht vollständig frei vereinbart werden.

\section{Grundlagen zu Softwarelizenzen}
Im Gegensatz zu technischen Erfindungen die meistens durch Patente und Gebrauchsmuster geschützt werden, sind Computerprogramme durch das Urheberrecht gemäß §2 UrhG geschützt,
wobei die Idee hinter einem Computerprogramm durch ein Patent geschützt werden kann. So ist der besondere Schutz von Computerprogrammen in §§69a bis 69g UrhG genauer beschrieben,
aber schränkt auch den Urheberrechtsschutz ein, indem erklärt wird, dass durch den Urheberrechtsschutz “Ideen und Grundsätze, die einem Element eines Computerprogramms zugrunde liegen, einschließlich der den Schnittstellen zugrundeliegenden Ideen und Grundlagen,
[...] nicht geschützt [sind]” (§69a Abs.2 UrhG).  Des Weiteren muss ein Computerprogramm, um geschützt zu sein, laut Gesetz ein “individuelle[s] Werk[e] in dem Sinne darstellen,
dass [es] das Ergebnis der eigenen geistigen Schöpfung [seines] Urhebers [ist]” (§69a Abs. 3 UrhG). Somit ist es meistens der Fall, dass eher die Schutzfähigkeit statt die Schutzunfähigkeit bei Computerprogrammen die Regel darstellt.
\seFootcite{vgl.}{S. 5 ff.}{SVH:ITR}

Gegenstände des Schutzes sind laut §69a Abs. 1 “Programme in jeder Gestalt, einschließlich des Entwurfsmaterials”. Der Gesetzgeber schützt somit sowohl den Quellcode,
als auch den Bytecode eines Programms, unabhängig davon in welcher Programmiersprache die Software erstellt wurde, wobei es einige Ausnahmen, wie zum Beispiel Benutzeroberflächen, gibt,
die nicht unter das Schutzrecht nach §69a UrhG fallen, aber durch das Designrecht geschütz werden können. Anders als bei Patenten entstehen Urheberrechte ohne ein Anmeldeverfahren und auch vor der Veröffentlichung des Werkes,
sodass vor Allem bei Computerprogrammen häufig Doppelschöpfungen auftreten, bei denen mindestens zwei Entwickler den nahezu gleichen Quellcode programmieren.
Besteht eine solche Doppelschöpfung, so haben beide Entwickler das Urheberrecht auf ihre Programme und können sich nicht gegenseitig die Nutzung untersagen. \seFootcite{vgl.}{S. 9 ff.}{SVH:ITR}

Das Urheberrecht besteht aus drei Teilen: das Urheberpersönlichkeitsrecht (gemäß §§ 12-14 UrhG), das Verwertungsrecht (gemäß §15-24 UrhG), sowie sonstige Rechte (gemäß §§25-27 UrhG).
Zu dem Urhberpersönlichkeitsrecht gehört unter Anderem das Veröffentlichungsrecht gemäß §12 UrhG, das dem Urheber das Recht gibt, zu bestimmen ob sein Werk veröffentlicht wird oder nicht,
sowie “den Inhalt seines Werkes öffentlich mitzuteilen oder zu beschreiben, solange weder das Werk noch der wesentliche Inhalt oder eine Beschreibung des Werkes mit seiner
Zustimmung veröffentlicht ist.” Hingegen beinhaltet das Verwertungsrecht unter Anderem das Recht zur Vervielfältigung und Verbreitung des Werkes gemäß §§16, 17 UrhG.

Ist das Urheberrecht mit der Schöpfung des Werkes entstanden, so bleibt der Urheberrechtsschutz während der Lebenszeit des Urhebers sowie bis 70 Jahre nach seinem Tod bestehen (§§64 ff. UrhG).
Des Weiteren ist es ist in Deutschland mit Ausnahme der Vererbung nicht möglich diese Urheberrechte vollständig zu übertragen (§§28, 29 Abs. 1 UrhG). Jedoch können gemäß §§29 Abs. 2, 31 ff. UrhG
“Nutzungsrechte [eingeräumt werden sowie] schuldrechtliche Einwilligungen und Vereinbarungen zu Verwertungsrechten” getroffen werden. Eine Übertragung der Urheberpersönlichkeitsrechte
kann es, wie oben angedeutet, nur mittels Vererbung gemäß §28 UrhG geben.\seFootcite{vgl.}{S. 8}{SVH:ITR}

Um Arbeitgebern eine möglichst einfache Verwertung der Computerprogramme seiner Angestellten zu gewährleisten, legt der Gesetzgeber fest, dass der Arbeitgeber
“ausschließlich zur Ausübung aller vermögensrechtlichen Befugnisse an dem Computerprogramm berechtigt [ist], sofern nichts anderes vereinbart ist.” Der Gesetzgeber
sieht dabei das Gehalt als finanziellen Ausgleich für die Einräumung der Nutzungs- und Ververtungsrechte. \seFootcite{vgl.}{S. 11 ff.}{SVH:ITR}

Ähnlich wie bei den gewerblichen Schutzrechten, wie beispielsweise dem Patentrecht, ist beim Urheberrecht die Einräumung von Nutzungsrechten nach §§31 ff. UrhG und im Besonderen
für Software zusätzlich nach §§69a-69g UrhG möglich, wobei des Urheberrecht, wie bereits beschrieben, wegen des Urheberpersönlichkeitsrechts nicht vollständig übertragen werden kann.
Dennoch genießt die Übertragung von Nutzungsrechten von Software die im Rahmen des EG-Vertrages und des Gesetzes gegen Wettbewerbsbeschränkung erlaubte Vertragsfreiheit. Somit kann der Einsatzort, die Laufzeit,
die Art der Nutzung, sowie aber auch das Recht zur Abänderung, Weiterverarbeitung und Verbreitung vereinbart werden. So beinhalten beispielsweise Open-Source-Lizenzen die
Einsicht in den Quellcode des Programms, sowie das Recht zur Veränderung und Verbreitung. Im Gegensatz dazu beinhalten proprietäre Lizenzen eher kein Recht auf Einsicht in den
Quellcode oder gar Rechte zur Veränderung oder Verbreitung der Software, dahingegen aber häufig eine Vergütung als Gegenleistung für die Einräumung der Nutzungsrechte.
\seFootcite{vgl.}{S. 19 ff.}{SVH:ITR}

Neben dem Lizenzvertrag der zum Beispiel zwischen zwei Unternehmen geschlossen wird, gibt es häufig zusätzlich für den Endbenutzer sogenannte \gls{eula}.
Diese müssen meistens vor der Installation akzeptiert werden, um die Software letztendlich nutzen zu können.
Jedoch sind \gls{eula} in den meisten Fällen nicht rechtswirksam, weil \gls{eula} als AGBs des Softwareherstellers zu verstehen sind und auf diese AGBs bereits vor dem
Kauf ausdrücklich aufmerksam gemacht werden muss. So ist es nicht rechtmäßig, wenn ein Kunde nachdem er einen Vertrag mit dem Softwarehersteller über die Nutzung der Software vereinbart hat,
noch zusätzliche \gls{eula} akzeptieren muss, um letztendlich die Nutzungsrechte anzuwenden.\seFootcite{vgl.}{S. 72 ff.}{AB:EULA} \seFootcite{vgl.}{S. 72 ff.}{SVH:ITR}


\chapter{Gebrauchte Softwarelizenzen}
Gekaufte Software beziehungsweise die gekauften Lizenzen zur Benutzung einer Software sind nach dem Kauf als immaterielles Wirtschaftsgut beim Käufer unter Anderem
auf dem Konto EDV-Software verbucht und binden damit Kapital in Vermögen. So stellt sich für den Käufer die Frage, ob er dieses gebundende Kapital wieder liquidieren kann.
An dieser Stelle wird die Frage aufgeworfen, wie es juristisch möglich ist gekaufte Softwarelizenzen als gebrauchte Softwarelizenzen weiter zu verkaufen. \seFootcite{vgl.}{}{GS:Warum}

Gebrauchte Software entsteht in vielen Szenarien, wie beispielsweise bei einer Insolvenz oder wenn ein Unternehmen die Software wechselt und die alte Software nicht mehr benötigt.
Des Weiteren bietet das Kaufen von gebrauchten Softwarelizenzen auch für den Käufer Vorteile, wie zum Beispiel, dass gebrauchte Software zu einem geringeren Preis im Vergleich
zu dem ungebrauchten Äquivalent verkauft wird oder dass ältere Versionen einer Software eventuell nicht mehr vom Hersteller verkauft wird. \seFootcite{vgl.}{}{GS:Warum}

Jedoch wird häufig darüber gestritten, ob ein Weiterverkauf von Softwarelizenzen legitim ist, wenn der Hersteller, wie es häufig der Fall ist, eine Weiterverbreitung der
Softwarelizenz im Lizenzvertrag untersagt. Dagegen steht allerdings §34 Abs. 1 UrhG der besagt, dass “ein Nutzungsrecht [...] nur mit Zustimmung des Urhebers übertragen werden [kann]”,
und weiter, dass “der Urheber [...] die Zustimmung nicht wider Treu und Glauben verweigern [darf].” Das bedeutet, dass der Urheber ohne Angabe eines wichtigen
Grundes die Zustimmung zur weiteren Übertragung der Nutzungsrechte nicht verweigern kann. Des Weiteren gilt der Erschöpfungsgrundsatz bei Computerprogrammen gemäß §69 c Nr. 3 Satz 2,
der erklärt, dass wenn “ein Vervielfältigungsstück eines Computerprogramms mit Zustimmung des Rechtsinhabers im Gebiet der Europäischen Union [...] im Wege der Veräußerung in Verkehr
gebracht [wird], so erschöpft sich das Verbreitungsrecht in [B]ezug auf dieses Vervielfältigungsstück”. Daraus folgt, dass wenn ein Unternehmen eine Software gekauft hat und diese
nicht mehr benötigt, sie dieses Vervielfältigungsstück weiter verbreiten und somit verkaufen können. \seFootcite{vgl.}{}{GS:Gesetze}

Bezüglich gebrauchter Software gab es einige richtungsweisende Urteile, die die oben genannten Gesetze näher interpretieren. So klagte die Microsoft Coorporation
gegen die Weiterveräußerung der von ihr entwickelten und verkauften \gls{oem}. \gls{oem} bedeutet,
dass die Software an eine bestimmte Hardware, die mit ausgeliefert wird, gebunden ist. Der Angeklagte in diesem Fall veräußerte nach dem Kauf die \gls{oem} weiter.
Der \gls{bgh} entschied sich in ihrem Urteil vom 06.07.2000 die Klage abzulehnen, da “die Programmversion durch den Hersteller oder mit seiner Zustimmung in Verkehr gesetzt worden
[und somit] die Weiterverbreitung aufgrund der eingetretenen Erschöpfung des urheberrechtlichen Verbreitungsrechts ungeachtet einer inhaltlichen Beschränkung des eingeräumten
Nutzungsrechts frei” ist (gemäß § 69c Nr. 3 Satz 2, § 17 Abs. 2 und § 32 UrhG).\seFootcite{}{}{BGH:micro}

Ein weiteres entscheidendes Urteil wurde am 03.07.2012 vom EuGH getroffen.\seFootcite{}{}{EuGH:oracle} Dabei klagte Oracle, die ihre sogenannte “Client-Server-Software” per Download mit zusätzlichen
Softwarepflegeverträgen verkauft, gegen UsedSoft, die die Software Oracle-Kunden abgekauft und weiterveräußert hatte. Das Urteil des EuGH lautete, “dass der Grundsatz der
Erschöpfung des Verbreitungsrechts nicht nur dann gilt, wenn der Urheberrechtsinhaber die Kopien seiner Software auf einem Datenträger (CD-ROM oder DVD) vermarktet, sondern auch dann,
wenn er sie durch Herunterladen von seiner Internetseite verbreitet.” Des Weiteren ist aber auch der Erstkäufer verpflichtet, nach der Weiterveräußerung alle Softwarekopien
einschließlich Handbücher und so weiter zu löschen beziehungsweise dem Zweitkäufer  zu übergeben. “Außerdem erstreckt sich die Erschöpfung des Verbreitungsrechts auf die
Programmkopie in der vom Urheberrechtsinhaber verbesserten und aktualisierten Fassung“ und somit also auf die Updates der Software von Oracle.\seFootcite{vgl.}{}{CW:Oracle}


\chapter{Kriterien zum Vergleich von Softwarelizenzen}
% For reference: http://www.gnu.org/licenses/ 
Nachdem nun auf properitäre Software und deren Weiterverkauf eingegangen wurde sollen nun Open Source Lizenzen betrachtet werden. Welche Lizenzen kann ein Softwareentwickler w\"ahlen, wenn er sein Programm der Allgemeinheit zur Verf\"ugung stellen will. \todo{Passt der Satz hier???} In diesem Kapitel wird erl\"autert welche Kriterien f\"ur einen Vergleich der Lizenzen herangezogen werden und warum diese Kriterien Relevanz f\"ur Softwareentwickler besitzen.

Der Softwareentwickler w\"ahlt vor dem Vertrieb seiner Software eine Lizenz aus, unter der er seine Software ver\"offentlichen will. Mit der Lizenz wird geregelt, welche Nutzungsrechte f\"ur die Software einger\"aumt werden. F\"ur die Wahl der Lizenz muss der Entwickler also entscheiden was mit seiner Software getan werden darf. Dies ist vor allem bei Open Source interessant, da die Offenlegung des Quellcodes dem Nutzer mehr M\"oglichkeiten bieten soll. Um diese M\"oglichkeiten zu erlauben, m\"ussen die Rechte in einer Lizenz einger\"aumt werden, da andere sonst gegen die Urheberrechte verstoßen k\"onnten. Was erlaubt wird und was nicht und unter welchen Bedingungen beispielsweise Weiterentwicklungen erm\"oglicht werden, wird durch die Lizenz gesteuert. 

(https://www.gnu.org/philosophy/free-sw) Nach der GNU Philosophie muss freie Software dem Nutzer vier wesentliche Freiheiten gew\"ahren. Wenn diese Freiheiten gew\"ahrt werden, kann die Software als freie Software bezeichnet werden, “die die Freiheit und Gemeinschaft der Nutzer respektiert”. 

(https://www.gnu.org/philosophy/free-sw) 
% (\citealt[][S.\,0]{key0000})
\begin{seList} 
\item Die Freiheit, das Programm auszuf\"uhren wie man m\"ochte, f\"ur jeden Zweck (Freiheit 0).
\item Die Freiheit, die Funktionsweise des Programms zu untersuchen und eigenen Datenverarbeitungbed\"urfnissen anzupassen (Freiheit 1). Der Zugang zum Quellcode ist daf\"ur Voraussetzung.
\item Die Freiheit, das Programm weiterzuverbreiten und damit seinen Mitmenschen zu helfen (Freiheit 2).
\item Die Freiheit, das Programm zu verbessern und diese Verbesserungen der \"Offentlichkeit freizugeben, damit die gesamte Gemeinschaft davon profitiert (Freiheit 3). Der Zugang zum Quellcode ist daf\"ur Voraussetzung.
\end{seList}
Diese Freiheiten werden als Kriterium “freie Software” in den Vergleich eingehen. Nur wenn die vier Freiheiten erf\"ullt sind, kann eine Lizenz als Lizenz f\"ur freie Software gewertet werden. 

(http://www.gnu.org/licenses/copyleft.de.html) Die Offenlegung des Quellcodes erm\"oglicht Weiterentwicklungen und Modifizierungen am Programm. In welcher Form diese weiterverbreitet werden k\"onnen, kann ebenfalls durch die Lizenz geregelt werden. In diesem Zusammenhang ist die Betrachtung des sogenannte Copyleft interessant. Copyleft ist im Prinzip die Umkehrung des Copyrights, welches das Werk vor \"Anderungen sch\"utzt. Mit dem Copyleft wird das Recht den “Quellcode des Programms [...] zu nutzen, zu modifizieren und weiterzuverarbeiten [nur dann einger\"aumt], wenn die Vertriebsbedingungen unver\"andert bleiben.” Es wird also sichergestellt, dass zuk\"unftige Weiterentwicklungen nicht Properit\"ar werden k\"onnen. 

(http://www.heise.de/open/artikel/Open-Source-Lizenzen-221957.html ) Das Copyleft kann zus\"atzlich noch in starkes Copyleft und schwaches Copyleft unterschieden werden. Das starke Copyleft verbietet es properit\"arer Software gegen das Programm / die Bibliothek zu linken. Das heißt es werden sogar Aufrufe durch fremde Software, die nicht unter den gleichen Vertriebsbedingungen stehen verhindert. Das schwache Copyleft erlaubt es - im Gegensatz zum starken Copyleft - dass properit\"arer Code gegen das Programm linken darf. Das bedeutet Schnittstellen des Programms k\"onnen verwendet werden. In beiden F\"allen werden jedoch Modifikationen und Weiterentwicklungen am Programm / der Bibliothek selbst gesch\"utzt, sodass hier die Vertriebsbedingungen gleich bleiben. 

(http://www.oreilly.de/catalog/gplger/chapter/ch02.pdf) Ob das Linken bei starkem Copyleft erlaubt ist oder nicht ist davon abh\"angig, ob das Programm als abgeleitetes Werk bezeichnet werden kann. “Es ist daher notwendig, inhaltlich und funktional zu bewerten, ob zwei Softwarebestandteile eine Einheit bilden oder ob ihnen selbstst\"andige und unabh\"angige Funktionen zukommen. Im Einzelfall kann dies zu schwierigen Auslegungsfragen f\"uhren, die sich teil nicht immer zweifelsfrei beantworten lassen.“

(https://sfconservancy.org/news/2015/oct/28/vmware-update/) Wie die Auslegung des starken Copylefts ausgestaltet wird, kann in einem laufenden Verfahren gegen VMware beobachtet werden. Die Verhandlung ist f\"ur das erste Quartal 2016 angesetzt. Gekl\"art werden muss, ob Software Bestandteile unter GPL gestellt werden m\"ussen oder nicht, wenn sie gegen Teile des Linux Kernels linken. Im konkreten Fall handelt es sich um Kernelmodule die im gleichen Kontext laufen, wie der Copyleft lizenzierte Programmcode.

Die meisten Open Source-Lizenzen enthalten Haftungsausschl\"usse, da Open Source Software unentgeltlich zur Verf\"ugung gestellt wird und die Entwickler deshalb eine Haftung ablehnen. (ISBN 3-89842-606-8 Seite 134f) In Deutschland sind diese Haftungsauschl\"usse jedoch unwirksam, es sei denn der Entwickler bietet die Software vollst\"andig kostenlos an, da in diesem Fall die Regelungen eines Schenkungsvertrags Anwendung finden. (http://blog-it-recht.de/2013/01/24/softwarentwickler-haftung-fur-open-source-software-auf-basis-der-gpl/). Da die Gew\"ahrleistungs- und Haftungsauslussklauseln in Deutschland unwirksam sind, sollen sie nicht untersucht werden.

Im nachfolgenden Vergleich wird also einerseits darauf eingegangen ob es sich um eine freie Lizenz nach der Definition der Free Software Foundation handelt und auf der Seite ob es sich um eine Lizenz mit Copyleft handelt. 

\chapter{Vergleich der Lizenzen}
\section{Wahl der Lizenzen}
In diesem Vergleich sollen Open Source Lizenzen verglichen werden, um aus Entwickler Perspektive die Wahl einer Lizenz zu erleichtern. Im letzten Abschnitt wurden die Kriterien erl\"autert, anhand derer die Lizenzen verglichen werden. 

Es sollen dabei bekannte Lizenzen verglichen werden, da die Verwendung bekannter Open Source Lizenzen daf\"ur sorgt, dass deren Auswirkungen bereits bekannt sind und man nicht die gesamten Lizenzbedingungen lesen muss. 

Aus diesem Grund werden die f\"unf am h\"aufigsten genutzten Lizenzen von der Code-Sharing Plattform Github verglichen. 

(https://github.com/blog/1964-open-source-license-usage-on-github-com) Diese sind die MIT-Lizenz, die GPLv2 Lizenz, die Apache Lizenz, die GPLv3 Lizenz und die BSD 3-clause. 

\section{X11 Lizenz (MIT Lizenz)}

(http://opensource.org/licenses/MIT) Die MIT-Lizenz erlaubt das Kopieren, Kombinieren, Vertreiben, Unterlizenzieren und den Verkauf der Software solange der Copyright Hinweis und der Lizenztext mitgeliefert werden. 

Die vier Freiheiten werden einger\"aumt, da die Lizenz jegliche Nutzung, Weiterentwicklung und das Weiterverbreiten erlaubt. Auch das Verbreiten von Weiterentwicklungen wird durch diese Lizenz erlaubt. 

Die Lizenz besitzt kein Copyleft, der weitere Vertrieb des Codes wird nicht durch die Lizenz vorgeschrieben, lediglich der Copyright Hinweis und der Lizenztext m\"ussen mitgeliefert werden. 

\section{GPLv2 und GPLv3 Lizenz}

(http://www.gnu.org/licenses/old-licenses/gpl-2.0 und http://www.gnu.org/licenses/gpl) Die GPL Lizenz wurde entwickelt um die Freiheit Software zu teilen und zu ver\"andern f\"ur alle Nutzer zu garantieren. Die GPL erlaubt das Kopieren, Verteilen und Ver\"andern der Software. Das Kopieren und Verteilen der Software ist allerdings an die Bedingung gebunden, dass die Empf\"anger der Software die gleichen Rechte besitzen muss wie man selbst, dass man ihnen den Quellcode zur Verf\"ugung stellen muss und den Nutzern die Bedingungen zeigt, sodass diese \"uber ihre Rechte informiert sind. 

(http://www.gnu.org/licenses/gpl) Soll eine ver\"anderte Version des Programms weitergegeben werden, so muss es nach Ziffer 5 der Lizenzbedingungen auff\"allige Vermerke (“prominent notices”) tragen dass es ver\"andert wurde. Außerdem muss es an jeden lizenziert werden, der eine Kopie des Programms erlangt. Besitzt das Programm ein User Interface m\"ussen dort die rechtlichen Hinweise angezeigt werden.

(http://www.gnu.org/licenses/gpl) Die \"Ubertragung des Programms ist nach Ziffer 6 der Lizenzbedingungen nur dann erlaubt, wenn der Quelltext in einer maschinen-lesbaren Quelle zur Verf\"ugung gestellt wird. 

(http://www.gnu.org/licenses/rms-why-gplv3) Die GPLv2 und die GPLv3 unterscheiden sich in ihren Zielen nicht. Bei der GPLv3 handelt es sich um eine Weiterentwicklung. Beispielsweise wird die “Tivoization” verhindert. Unter “Tivoization” wird das Verhindern von \"Anderungen am Programm auf Endger\"aten verstanden, durch zur\"uckhalten von Schl\"usseln, Methoden oder anderen Informationen die n\"otig w\"aren. Da dies die Freiheit des Nutzers einschr\"ankt, wird durch die GPLv3 gew\"ahrleistet, dass dem Nutzer die M\"oglichkeit gegeben werden muss. 

(http://www.gnu.org/licenses/gpl) In Ziffer 6 der GPL wird gefordert, dass f\"ur “User Products” alle Methoden, Prozeduren, Authorisierungs Schl\"ussel und weiteren n\"otigen Informationen zur Installation und Ausf\"uhrung von ver\"andertem Code mitgeliefert werden m\"ussen. 

(http://www.gnu.org/licenses/rms-why-gplv3) Ein weiterer Punkt der durch die GPLv3 verbessert wird ist, dass Software Patente die einen Teil der Benutzer von der Nutzung ausschr\"anken k\"onnen, nicht erlaubt werden. 

Die vier Freiheiten werden einger\"aumt, da die Lizenz die Nutzung, Weiterentwicklung und das Weiterverbreiten erlaubt. Auch das Verbreiten von Weiterentwicklungen wird durch diese Lizenz erlaubt.

Die Lizenz besitzt Copyleft, da beim weiteren Vertrieb des Codes die Rechte erhalten bleiben m\"ussen, dass heißt die Lizenz kann nicht ver\"andert werden. Es handelt sich um starkes Copyleft, da das Einbinden propriet\"arer Bestandteile untersagt ist. 

\section{Apache Lizenz}

(http://www.apache.org/licenses/LICENSE-2.0) Bei Projekten die unter der Apache2 Lizenz stehen gew\"ahrt jeder Beitragende nach Ziffer 2 eine unbefristete, weltweite, nicht ausschließliche, kostenlose, gebührenfreie, unwiderrufliche Urheberrechtslizenz zur Vervielfältigung, zur Bearbeitung, zur öffentlichen Ausstellung, Aufführung, Unterlizenzierung und Verbreitung des Werks und derartiger Bearbeitungen in Quell- oder Objektform. Ziffer 3 beinhaltet eine Klausel die \"Ahnlich der GPLv3 dem Schutz vor einschr\"ankenden Patenten dient. 

(http://www.apache.org/licenses/LICENSE-2.0) F\"ur die Weiterverbreitung muss nach Ziffer 4 beachtet werden, dass Kopien oder veränderte Kopien auf jedem Medium verbreitet werden d\"urfen, solange sie die folgenden Bedingungen einhalten. Jeder Empf\"anger muss eine Kopie der Lizenz erhalten, ver\"anderte Dateien m\"ussen gekennzeichnet werden, Hinweise zu Urhebern d\"urfen aus der Quellform nicht entfernt werden und sofern eine NOTICE Datei existiert muss diese mit verbreitet werden. 

Die vier Freiheiten werden einger\"aumt, da die Lizenz die Nutzung, Weiterentwicklung und das Weiterverbreiten erlaubt. Auch das Verbreiten von Weiterentwicklungen wird durch diese Lizenz erlaubt.

Die Lizenz besitzt kein Copyleft, da die Art des weiteren Vertrieb des Codes nicht durch die Lizenz vorgeschrieben wird. Es m\"ussen jedoch die genannten Bedingungen f\"ur eine Weiterverbreitung erf\"ullt werden.

\section{BSD 3-clause}

(https://opensource.org/licenses/BSD-3-Clause) Die Weiterverbreitung in Quellformat, mit oder ohne Ver\"anderung ist erlaubt, wenn der Copyright Hinweis, der Lizenztext und der Disclaimer weiterhin enthalten sind. In Bin\"arformat m\"ussen Copyright Hinweis, der Lizenztext und der Disclaimer in der Dokumentation und / oder in anderem Material welches dem Programm beiliegt enthalten sein. Weder der Name des Copyright haltenden noch eines Beitragenden d\"urfen ohne deren Erlaubnis f\"ur die Bewerbung des Produkts verwendet werden.

Die vier Freiheiten werden einger\"aumt, da die Lizenz die Nutzung, Weiterentwicklung und das Weiterverbreiten erlaubt. Auch das Verbreiten von Weiterentwicklungen wird durch diese Lizenz erlaubt.

Die Lizenz besitzt kein Copyleft, da die Art des weiteren Vertrieb des Codes nicht durch die Lizenz vorgeschrieben wird. Es m\"ussen jedoch die genannten Bedingungen f\"ur eine Weiterverbreitung erf\"ullt werden.

\section{Entscheidungshilfe f\"ur Entwickler}
\todo[inline]{Brauchen wir so ein Kapitel? Die Unterschiede die ich festgestellt habe, sind minimal. Alle Lizenzen sind "Freie Software". Eine davon ist eine starke Copyleft; die anderen vier sind ohne Copyleft}

\chapter{Zusammenfassung und Ausblick}
% Ziel dieser Arbeit war es, die rechtlichen Probleme von Software Lizenzen zu betrachten und Open Source Lizenzen zu vergleichen. 

Die fünf am weitesten verbreiteten Open Source Lizenzen auf Github gew\"ahren jedem Nutzer die vier Freiheiten um als freie Software zu gelten. Somit gilt f\"ur die meiste ver\"offentlichte Software, dass man sie ausf\"uhren, untersuchen, weiterverbreiten und verbessern darf. Bei freier Software kann der Nutzer das Programm kontrollieren, im Gegensatz zur properitären Software, bei der der Eigentümer die Kontrolle \"uber die Funktionsweise hat. 

%=================================================================





% Anhang der Arbeit
%
%

% Der Anhang sollte auf einer neuen Seite beginnen; daher wird der Seitenvorschub bei neuen Kapiteln
% wieder angeschaltet; Achtung: die Verwendung von newpage erzeugt eine Kopfzeile, was dann nicht zu dem
% Gesamtlayout des Dokuments passt
%
%
%\seChaptersNewpage
%\seAppendix{}

%\input{\seWaPathText/se-anhang}

%
%  Erzeugung eines Glossars
%
% Achtung: Das Glossar wird nur ausgegeben, wenn mindestens ein Eintrag in der Arbeit
%                definiert wurde
%
%

% Die folgenden Kapitel beginnen jeweils auf einer neuen Seite
%
%
\seChaptersNewpage{}
%\newpage
\sePrintGlossary{}


%
% Literaturverzeichnisses
%
%\newpage
\sePrintBibliography{}

\input{\seWaPathText/se-test-literaturverzeichnis}


%
% Festlegung des grundlegenden Formatierungsstils des Literaturverzeichnis
%
\bibliographystyle{jurabib}

% Eigentliche Ausgabe der in der Arbeit verwendeten Quellen
%
%
% Angabe der bib-Dateien, in denen die Quellen beschrieben sind;
% die Angabe geht davon aus, dass eine wa.bib-Datei in demselben
% Verzeichnis liegt, wie se-ba-vorlage.tex
%

% 2012-02-06
%
% Umbenennung von Literatur- in Quellenverzeichnis
%
%\renewcommand*{\bibname}{Quellenverzeichnis}
\seBibliography{wa}


%
% Erzeugung der ehrenw\"ortlichen Erkl\"arung
%
% Der optionale Parameter kann verwendet werden, um f\"ur das Thema der Arbeit eine
% andere Formatierung vorzunehmen; das sollte in der Regel nicht erforderlich sein;
% ausserdem besteht die Gefahr inkonsistenter Titel auf dem Titelblatt und in der
% ehrenw\"ortlichen Erkl\"arung
%
\seEhrenwoertlicheErklaerung{} % dieses Kommando sollte standardm\"assig verwendet werden
%\seEhrenwoertlicheErklaerung[\LaTeX-Vorlage zur Anfertigung einer Seminararbeit (Version \version{})]


\end{document}
