\section{Fu{\ss}noten}

\subsection{Verwendung dreistelliger Fu{\ss}noten}

Bei dreistelligen Fu{\ss}noten tritt das Problem auf, dass der Abstand zwischen Fu{\ss}notennummer und folgendem Text nicht mehr ausreicht.
Der Abstand kann wie folgt vergr\"o{\ss}ert werden:

\begin{seList}
\item
In der Style-Datei \texttt{se-jb-footmisc.sty} wird \"uber \newline \texttt{\textbackslash{}setlength\textbackslash{}footnotemargin\{0.3cm\}} genau dieser Abstand definiert.
\item
\"Andert man den Wert z.\,B.\ auf 0.5cm, dann sollte es auch f\"ur dreistellige Fu{\ss}noten ausreichen.
\end{seList}

\subsection{Fu{\ss}noten in Abbildungen, Tabellen und Programmlistings}

\LaTeX{} erlaubt generell nicht die Verwendung des Kommandos \verb+\footnote+ in \textsl{Gleitobjekten} (\textsl{Floats}). 
Zu den Gleitobjekten geh\"oren \textsl{Abbildungen}, \textsl{Tabellen} und auch \textsl{Programmlistings}. In \vref{fussnote} findet ein 
kleiner \textsl{Workaround} Anwendung, wie man doch Fu{\ss}noten in Gleitobjekten angeben kann. 

\begin{seList}
\item Mit \verb+\footnotemark+ wird in dem \verb+\caption+-Kommando die \textsl{Fu{\ss}notennummer} erzeugt.
\item Mit dem Kommando \verb+\footnotetext+ wird au{\ss}erhalb der Umgebung, die das Gleitobjekt definiert (z.\,B. die 
\verb+figure+-Umgebung), der Text der Fu{\ss}note festgelegt. Hierbei ist zu beachten, dass ein Gleitobjekt auf die n\"achste Seite 
verschoben werden kann. In einem derartigen Fall sollte der Fu{\ss}notentext an einer Stelle im \LaTeX-Quelltext positioniert werden, die ebenfalls zu dieser Seite geh\"ort.\footnote{Standardm\"a{\ss}ig wird man \texttt{\textbackslash{}footnotetext} direkt hinter dem Gleitobjekt definieren, um sicherzustellen,
dass der Fu{\ss}notentext auch der richtigen Fu{\ss}notennummer zugeordnet wird.}
\end{seList}