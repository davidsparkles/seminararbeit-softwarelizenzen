Im Gegensatz zu technischen Erfindungen die meistens durch Patente und Gebrauchsmuster geschützt werden, sind Computerprogramme durch das Urheberrecht gemäß §2 UrhG geschützt,
wobei die Idee hinter einem Computerprogramm auch zusätzlich durch ein Patent geschützt werden kann. Der besondere Schutz von Computerprogrammen wird in §§69a bis 69g UrhG, in denen erklärt wird,
dass durch den Urheberrechtsschutz “Ideen und Grundsätze, die einem Element eines Computerprogramms zugrunde liegen, einschließlich der den Schnittstellen zugrundeliegenden Ideen und Grundlagen,
[...] nicht geschützt [sind]” (§69a Abs.2 UrhG).  Des Weiteren muss ein Computerprogramm, um geschützt zu sein, laut Gesetz ein “individuelle[s] Werk[e] in dem Sinne darstellen,
dass [es] das Ergebnis der eigenen geistigen Schöpfung [seines] Urhebers [ist]” (§69a Abs. 3 UrhG). Somit ist es meistens der Fall, dass eher die Schutzfähigkeit statt die Schutzunfähigkeit die Regel darstellt.
\seFootcite{vgl.}{S. 5 ff.}{SVH:ITR}

Gegenstände des Schutzes sind laut §64a Abs. 1 “Programme in jeder Gestalt, einschließlich des Entwurfsmaterials”. Der Gesetzgeber schützt somit sowohl den Quellcode,
als auch den Bytecode eines Programms, unabhängig davon in welcher Programmiersprache die Software erstellt wurde, wobei es einige Ausnahmen, wie zum Beispiel Benutzeroberflächen, gibt,
die nicht unter das Schutzrecht nach §69a UrhG fallen. Anders als bei Patenten entstehen Urheberrechte ohne ein Anmeldeverfahren und auch vor der Veröffentlichung des Werkes,
sodass vor Allem bei Computerprogrammen häufig zu Doppelschöpfungen auftreten, bei denen mindestens zwei Entwickler den nahezu gleichen Quellcode programmieren.
Besteht eine solche Doppelschöpfung, so haben beide Entwickler das Urheberrecht auf ihre Programme und können sich nicht gegenseitig die Nutzung untersagen. \seFootcite{vgl.}{S. 9 ff.}{SVH:ITR}

Das Urheberrecht teilt sich in die drei Teile: Urheberpersönlichkeitsrecht (gemäß §§ 12-14 UrhG), Verwertungsrecht (gemäß §15-24 UrhG), sowie sonstige Rechte (gemäß §§25-27 UrhG).
Zu dem Urhberpersönlichkeitsrecht gehört unter Anderem das Veröffentlichungsrecht gemäß §12 UrhG, das dem Urheber das Recht gibt, zu bestimmen ob sein Werk veröffentlicht wird oder nicht,
sowie “den Inhalt seines Werkes öffentlich mitzuteilen oder zu beschreiben, solange weder das Werk noch der wesentliche Inhalt oder eine Beschreibung des Werkes mit seiner
Zustimmung veröffentlicht ist.” Hingegen beinhaltet das Verwertungsrecht unter Anderem das Recht des Urhebers zur Vervielfältigung und Verbreitung seines Werkes gemäß §§16, 17 UrhG.

Ist das Urheberrecht mit der Schöpfung des Werkes entstanden, so bleibt der Urheberrechtsschutz während der Lebenszeit des Urhebers sowie bis 70 Jahre nach seinem Tod bestehen (§§64 ff. UrhG).
Des Weiteren ist es ist in Deutschland mit Ausnahme der Vererbung nicht möglich diese Urheberrechte vollständig zu übertragen (§§28, 29 Abs. 1 UrhG). Jedoch können gemäß §§29 Abs. 2, 31 ff. UrhG
“Nutzungsrechte [eingeräumt werden sowie] schuldrechtliche Einwilligungen und Vereinbarungen zu Verwertungsrechten” getroffen werden. Eine Übertragung der Urheberpersönlichkeitsrechte
kann es somit nur mittels Vererbung gemäß §28 UrhG geben.\seFootcite{vgl.}{S. 8}{SVH:ITR}

Um Arbeitgebern eine möglichst einfache Verwertung der Computerprogramme seiner Angestellten zu gewährleisten, legt der Gesetzgeber fest, dass der Arbeitgeber
“ausschließlich zur Ausübung aller vermögensrechtlichen Befugnisse an dem Computerprogramm berechtigt [ist], sofern nichts anderes vereinbart ist.” Der Gesetzgeber
sieht dabei das Gehalt als finanziellen Ausgleich für die Einräumung der Nutzungs- und Ververtungsrechte. \seFootcite{vgl.}{S. 11 ff.}{SVH:ITR}

Ähnlich wie bei den gewerblichen Schutzrechten, wie beispielsweise dem Patentrecht, ist beim Urheberrecht die Einräumung von Nutzungsrechten nach §§31 ff. UrhG und im Besonderen
für Software zusätzlich nach §§69a-69g UrhG möglich, wobei des Urheberrecht, wie bereits beschrieben, wegen des Urheberpersönlichkeitsrechts nicht vollständig übertragen werden kann.
Dennoch genießt die Übertragung von Nutzungsrechten von Software die im Rahmen des EG-Vertrages und des Gesetzes gegen Wettbewerbsbeschränkung erlaubte Vertragsfreiheit. Somit kann der Einsatzort, die Laufzeit,
die Art der Nutzung, sowie aber auch das Recht zur Abänderung, Weiterverarbeitung und Verbreitung vereinbart werden. So beinhalten beispielsweise Open-Source-Lizenzen die
Einsicht in den Quellcode des Programms, sowie das Recht zur Veränderung und Verbreitung. Im Gegensatz dazu beinhalten proprietäre Lizenzen eher kein Recht auf Einsicht in den
Quellcode oder gar Rechte zur Veränderung oder Verbreitung der Software, dahingegen aber häufig eine Vergütung als Gegenleistung für die Einräumung der Nutzungsrechte.
\seFootcite{vgl.}{S. 19 ff.}{SVH:ITR}

Neben dem Lizenzvertrag der zum Beispiel zwischen zwei Unternehmen geschlossen wird, gibt es häufig zusätzlich für den Endbenutzer sogenannte \gls{eula}.
Diese müssen meistens vor der Installation akzeptiert werden, um die Software letztendlich nutzen zu können.
Jedoch sind \gls{eula} in den meisten Fällen nicht rechtswirksam, weil \gls{eula} als AGBs des Softwareherstellers zu verstehen sind und auf diese AGBs ausdrücklich bereits vor dem
Kauf aufmerksam gemacht werden muss. So ist es nicht rechtmäßig, wenn ein Kunde nachdem er einen Vertrag mit dem Softwarehersteller über die Nutzung der Software vereinbart hat,
noch zusätzliche \gls{eula} akzeptieren muss, um letztendlich die Nutzungsrechte umzusetzen.\seFootcite{vgl.}{S. 72 ff.}{AB:EULA} \seFootcite{vgl.}{S. 72 ff.}{SVH:ITR}
