\section{Wahl der Lizenzen}
In diesem Vergleich sollen Open Source Lizenzen verglichen werden, um aus Entwickler Perspektive die Wahl einer Lizenz zu erleichtern. Im letzten Abschnitt wurden die Kriterien erl\"autert, anhand derer die Lizenzen verglichen werden. 

Es sollen dabei bekannte Lizenzen verglichen werden, da die Verwendung bekannter Open Source Lizenzen daf\"ur sorgt, dass deren Auswirkungen bereits bekannt sind und man nicht die gesamten Lizenzbedingungen lesen muss. 

Aus diesem Grund werden die f\"unf am h\"aufigsten genutzten Lizenzen von der Code-Sharing Plattform Github verglichen. 

%(https://github.com/blog/1964-open-source-license-usage-on-github-com) 
Diese sind die MIT-Lizenz, die GPLv2 Lizenz, die Apache Lizenz, die GPLv3 Lizenz und die BSD 3-clause. \seFootcite{vgl.}{}{BB:USAGE}

\section{X11 Lizenz (MIT Lizenz)}

% (http://opensource.org/licenses/MIT) 
Die MIT-Lizenz erlaubt das Kopieren, Kombinieren, Vertreiben, Ver\"andern, Unterlizenzieren und den Verkauf der Software solange der Copyright Hinweis und der Lizenztext mitgeliefert werden. \seFootcite{vgl.}{}{LC:MIT}

Die vier Freiheiten werden einger\"aumt, da die Lizenz jegliche Nutzung, Weiterentwicklung und das Weiterverbreiten erlaubt. Auch das Verbreiten von Weiterentwicklungen wird durch diese Lizenz erlaubt. 

Die Lizenz besitzt kein Copyleft, der weitere Vertrieb des Codes wird nicht durch die Lizenz vorgeschrieben, lediglich der Copyright Hinweis und der Lizenztext m\"ussen mitgeliefert werden. 

\section{GPLv2 und GPLv3 Lizenz}

% (http://www.gnu.org/licenses/old-licenses/gpl-2.0 und http://www.gnu.org/licenses/gpl) 
Die GPL Lizenz wurde entwickelt um die Freiheit Software zu teilen und zu ver\"andern f\"ur alle Nutzer zu garantieren. Die GPL erlaubt das Kopieren, Verteilen und Ver\"andern der Software. Das Kopieren und Verteilen der Software ist allerdings an die Bedingung gebunden, dass die Empf\"anger der Software die gleichen Rechte besitzen muss wie man selbst, dass man ihnen den Quellcode zur Verf\"ugung stellen muss und den Nutzern die Bedingungen zeigt, sodass diese \"uber ihre Rechte informiert sind. \seFootcite{vgl.}{}{LC:GNU2} \seFootcite{vgl.}{}{LC:GNU3}

% (http://www.gnu.org/licenses/gpl) 
Soll eine ver\"anderte Version des Programms weitergegeben werden, so muss es nach Ziffer 5 der Lizenzbedingungen auff\"allige Vermerke (\glqq{}prominent notices\grqq{}) tragen dass es ver\"andert wurde. Außerdem muss es an jeden lizenziert werden, der eine Kopie des Programms erlangt. Besitzt das Programm ein User Interface m\"ussen dort die rechtlichen Hinweise angezeigt werden. \seFootcite{vgl.}{}{LC:GNU3}

% (http://www.gnu.org/licenses/gpl) 
Die \"Ubertragung des Programms ist nach Ziffer 6 der Lizenzbedingungen nur dann erlaubt, wenn der Quelltext in einer maschinen-lesbaren Quelle zur Verf\"ugung gestellt wird. \seFootcite{vgl.}{}{LC:GNU3}

(http://www.gnu.org/licenses/rms-why-gplv3) Die GPLv2 und die GPLv3 unterscheiden sich in ihren Zielen nicht. Bei der GPLv3 handelt es sich um eine Weiterentwicklung. Beispielsweise wird die \glqq{}Tivoization\grqq{} verhindert. Unter \glqq{}Tivoization\grqq{} wird das Verhindern von \"Anderungen am Programm auf Endger\"aten verstanden, durch zur\"uckhalten von Schl\"usseln, Methoden oder anderen Informationen die n\"otig w\"aren. Da dies die Freiheit des Nutzers einschr\"ankt, wird durch die GPLv3 gew\"ahrleistet, dass dem Nutzer die M\"oglichkeit gegeben werden muss. 

%(http://www.gnu.org/licenses/gpl) 
In Ziffer 6 der GPL wird gefordert, dass f\"ur \glqq{}User Products\grqq{} alle Methoden, Prozeduren, Authorisierungs Schl\"ussel und weiteren n\"otigen Informationen zur Installation und Ausf\"uhrung von ver\"andertem Code mitgeliefert werden m\"ussen. \seFootcite{vgl.}{}{LC:GNU3}

% (http://www.gnu.org/licenses/rms-why-gplv3) 
Ein weiterer Punkt der durch die GPLv3 verbessert wird ist, dass Software Patente die einen Teil der Benutzer von der Nutzung ausschr\"anken k\"onnen, nicht erlaubt werden. \seFootcite{vgl.}{}{RS:UPDATE}

Die vier Freiheiten werden einger\"aumt, da die Lizenz die Nutzung, Weiterentwicklung und das Weiterverbreiten erlaubt. Auch das Verbreiten von Weiterentwicklungen wird durch diese Lizenz erlaubt.

Die Lizenz besitzt Copyleft, da beim weiteren Vertrieb des Codes die Rechte erhalten bleiben m\"ussen, dass heißt die Lizenz kann nicht ver\"andert werden. Es handelt sich um starkes Copyleft, da das Einbinden propriet\"arer Bestandteile untersagt ist. 

\section{Apache Lizenz}

% (http://www.apache.org/licenses/LICENSE-2.0) 
Bei Projekten die unter der Apache2 Lizenz stehen gew\"ahrt jeder Beitragende nach Ziffer 2 eine unbefristete, weltweite, nicht ausschließliche, kostenlose, gebührenfreie, unwiderrufliche Urheberrechtslizenz zur Vervielfältigung, zur Bearbeitung, zur öffentlichen Ausstellung, Aufführung, Unterlizenzierung und Verbreitung des Werks und derartiger Bearbeitungen in Quell- oder Objektform. Ziffer 3 beinhaltet eine Klausel die \"Ahnlich der GPLv3 dem Schutz vor einschr\"ankenden Patenten dient. \seFootcite{vgl.}{}{LC:APACHE}

% (http://www.apache.org/licenses/LICENSE-2.0) 
F\"ur die Weiterverbreitung muss nach Ziffer 4 beachtet werden, dass Kopien oder veränderte Kopien auf jedem Medium verbreitet werden d\"urfen, solange sie die folgenden Bedingungen einhalten. Jeder Empf\"anger muss eine Kopie der Lizenz erhalten, ver\"anderte Dateien m\"ussen gekennzeichnet werden, Hinweise zu Urhebern d\"urfen aus der Quellform nicht entfernt werden und sofern eine NOTICE Datei existiert muss diese mit verbreitet werden. \seFootcite{vgl.}{}{LC:APACHE}

Die vier Freiheiten werden einger\"aumt, da die Lizenz die Nutzung, Weiterentwicklung und das Weiterverbreiten erlaubt. Auch das Verbreiten von Weiterentwicklungen wird durch diese Lizenz erlaubt.

Die Lizenz besitzt kein Copyleft, da die Art des weiteren Vertrieb des Codes nicht durch die Lizenz vorgeschrieben wird. Es m\"ussen jedoch die genannten Bedingungen f\"ur eine Weiterverbreitung erf\"ullt werden.

\section{BSD 3-clause}

% (https://opensource.org/licenses/BSD-3-Clause) 
Die Weiterverbreitung in Quellformat, mit oder ohne Ver\"anderung ist erlaubt, wenn der Copyright Hinweis, der Lizenztext und der Disclaimer weiterhin enthalten sind. In Bin\"arformat m\"ussen Copyright Hinweis, der Lizenztext und der Disclaimer in der Dokumentation und / oder in anderem Material welches dem Programm beiliegt enthalten sein. Weder der Name des Copyright haltenden noch eines Beitragenden d\"urfen ohne deren Erlaubnis f\"ur die Bewerbung des Produkts verwendet werden. \seFootcite{vgl.}{}{LC:BSD}

Die vier Freiheiten werden einger\"aumt, da die Lizenz die Nutzung, Weiterentwicklung und das Weiterverbreiten erlaubt. Auch das Verbreiten von Weiterentwicklungen wird durch diese Lizenz erlaubt.

Die Lizenz besitzt kein Copyleft, da die Art des weiteren Vertrieb des Codes nicht durch die Lizenz vorgeschrieben wird. Es m\"ussen jedoch die genannten Bedingungen f\"ur eine Weiterverbreitung erf\"ullt werden.

% \section{Entscheidungshilfe f\"ur Entwickler}
% \todo[inline]{Brauchen wir so ein Kapitel? Die Unterschiede die ich festgestellt habe, sind minimal. Alle Lizenzen sind "Freie Software". Eine davon ist eine starke Copyleft; die anderen vier sind ohne Copyleft}