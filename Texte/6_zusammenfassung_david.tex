Das Schutzrecht von Software basiert auf dem Urhebergesetzbuches und ist somit im Gegensatz zu anderen Schutzrechten nicht eintragungs- oder veröffentlichungspflichtig.
Damit Dritte die Software verwenden können ist eine Einräumung von Nutzungsrechten seitens des Urhebers notwendig, wobei aber lediglich die Verwertungsrechte an der
Software und nicht die Urheberpersönlichkeitsrechte übertragen werden können.

Sollte gekaufte Software vom Erstkäufer nicht mehr benötigt oder verkauft werden sollte, so ist dies in Form von Gebrauchtsoftware möglich. Da der Weiterverkauf von
Software umstritten ist gab es einige richtungsweisende Gerichtsbeschlüsse, wie der des Europäischen Gerichtshof vom 03.07.2012 im Fall Oracle gegen Usedsoft, indem
entschieden wurde, dass ein Weiterverkauf von Software auch möglich ist, wenn diese Software auf keinem physischen Datenträger sondern über das Herunterladen der Software
übergeben wird. Auch ein Weiterverkaufsverbot innerhalb der AGBs ist nach einer Entscheidung des Hamburger Landesgerichts vom 25.10.2013 ungültig. Somit ist
die Weiterveräußerung von Software zunächst zulässig.
