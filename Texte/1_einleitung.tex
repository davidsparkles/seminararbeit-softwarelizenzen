Das geistige Eigentum ist ein Gut und nach dem deutschen Gesetz schützenswert. Das bedeutet, dass die Urheber beziehungsweise Erfinder geistigen Eigentums, je nach Art bestimmte Rechte
an dem Eigentum selbst haben und gegebenenfalls Dritte von der Verwendung dieses Eigentums ausschließen oder einschränken. Insbesondere, wenn solch ein immaterielles Gut
ein Wirtschaftsgut ist, spielt der Schutz des geistigen Eigentums eine wichtige Rolle.

So wird in der Arbeit die Frage beleuchtet, wie Schutzrechte an Software entstehen und wie Nutzungsrechte an Software in Form von Lizenzverträgen eingeräumt werden.
Des Weiteren erklärt diese Seminararbeit, wie aus juristischer Sicht der Weiterverkauf von Softwarelizenzen funktioniert sowie was die Unterschiede zwischen
verschiedenen Open-Source Lizenzen sind. Zu diesem Zweck werden zunächst Kriterien aufgestellt, mit Hilfe derer ein Vergleich erstellt wird.

Als Quellen werden unter Anderem das Urhebergesetzbuch, das Patentgesetzbuch und das Designgesetzbuch herangezogen, wobei
diese durch Gerichtsurteile des Bundesgerichtshof sowie des Europäischen Gerichtshofes ergänzt werden. Desweiteren sind die Internetseiten ausgewählter
Open-Source Lizenzen die Quelle für deren Vergleich.
