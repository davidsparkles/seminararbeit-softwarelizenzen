Gekaufte Software beziehungsweise die gekauften Lizenzen zur Benutzung einer Software sind nach dem Kauf als immaterielles Wirtschaftsgut beim Käufer unter Anderem
auf dem Konto EDV-Software verbucht und binden damit Kapital in Vermögen. So stellt sich für den Käufer die Frage, ob er dieses gebundende Kapital wieder liquidieren kann.
An dieser Stelle wird die Frage aufgeworfen, wie es juristisch möglich ist gekaufte Softwarelizenzen als gebrauchte Softwarelizenzen weiter zu verkaufen. \seFootcite{vgl.}{}{GS:Warum}

Gebrauchte Software entsteht in vielen Szenarien, wie beispielsweise bei einer Insolvenz oder wenn ein Unternehmen die Software wechselt und die alte Software nicht mehr benötigt.
Des Weiteren bietet das Kaufen von gebrauchten Softwarelizenzen auch für den Käufer Vorteile, wie zum Beispiel, dass gebrauchte Software zu einem geringeren Preis im Vergleich
zu dem ungebrauchten Äquivalent verkauft wird oder dass ältere Versionen einer Software eventuell nicht mehr vom Hersteller verkauft wird. \seFootcite{vgl.}{}{GS:Warum}

Jedoch wird häufig darüber gestritten, ob ein Weiterverkauf von Softwarelizenzen legitim ist, wenn der Hersteller, wie es häufig der Fall ist, eine Weiterverbreitung der
Softwarelizenz im Lizenzvertrag untersagt. Dagegen steht allerdings §34 Abs. 1 UrhG der besagt, dass “ein Nutzungsrecht [...] nur mit Zustimmung des Urhebers übertragen werden [kann]”,
und weiter, dass “der Urheber [...] die Zustimmung nicht wider Treu und Glauben verweigern [darf].” Das bedeutet, dass der Urheber ohne Angabe eines wichtigen
Grundes die Zustimmung zur weiteren Übertragung der Nutzungsrechte nicht verweigern kann. Des Weiteren gilt der Erschöpfungsgrundsatz bei Computerprogrammen gemäß §69 c Nr. 3 Satz 2,
der erklärt, dass wenn “ein Vervielfältigungsstück eines Computerprogramms mit Zustimmung des Rechtsinhabers im Gebiet der Europäischen Union [...] im Wege der Veräußerung in Verkehr
gebracht [wird], so erschöpft sich das Verbreitungsrecht in [B]ezug auf dieses Vervielfältigungsstück”. Daraus folgt, dass wenn ein Unternehmen eine Software gekauft hat und diese
nicht mehr benötigt, sie dieses Vervielfältigungsstück weiter verbreiten und somit verkaufen können. \seFootcite{vgl.}{}{GS:Gesetze}

Bezüglich gebrauchter Software gab es einige richtungsweisende Urteile, die die oben genannten Gesetze näher interpretieren. So klagte die Microsoft Coorporation
gegen die Weiterveräußerung der von ihr entwickelten und verkauften \gls{oem}. \gls{oem} bedeutet,
dass die Software an eine bestimmte Hardware, die mit ausgeliefert wird, gebunden ist. Der Angeklagte in diesem Fall veräußerte nach dem Kauf die \gls{oem} weiter.
Der \gls{bgh} entschied sich in ihrem Urteil vom 06.07.2000 die Klage abzulehnen, da “die Programmversion durch den Hersteller oder mit seiner Zustimmung in Verkehr gesetzt worden
[und somit] die Weiterverbreitung aufgrund der eingetretenen Erschöpfung des urheberrechtlichen Verbreitungsrechts ungeachtet einer inhaltlichen Beschränkung des eingeräumten
Nutzungsrechts frei” ist (gemäß § 69c Nr. 3 Satz 2, § 17 Abs. 2 und § 32 UrhG).\seFootcite{}{}{BGH:micro}

Ein weiteres entscheidendes Urteil wurde am 03.07.2012 vom EuGH getroffen.\seFootcite{}{}{EuGH:oracle} Dabei klagte Oracle, die ihre sogenannte “Client-Server-Software” per Download mit zusätzlichen
Softwarepflegeverträgen verkauft, gegen UsedSoft, die die Software Oracle-Kunden abgekauft und weiterveräußert hatte. Das Urteil des EuGH lautete, “dass der Grundsatz der
Erschöpfung des Verbreitungsrechts nicht nur dann gilt, wenn der Urheberrechtsinhaber die Kopien seiner Software auf einem Datenträger (CD-ROM oder DVD) vermarktet, sondern auch dann,
wenn er sie durch Herunterladen von seiner Internetseite verbreitet.” Des Weiteren ist aber auch der Erstkäufer verpflichtet, nach der Weiterveräußerung alle Softwarekopien
einschließlich Handbücher und so weiter zu löschen beziehungsweise dem Zweitkäufer  zu übergeben. “Außerdem erstreckt sich die Erschöpfung des Verbreitungsrechts auf die
Programmkopie in der vom Urheberrechtsinhaber verbesserten und aktualisierten Fassung“ und somit also auf die Updates der Software von Oracle.\seFootcite{vgl.}{}{CW:Oracle}
