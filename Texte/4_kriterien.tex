% For reference: http://www.gnu.org/licenses/ 
Nachdem nun auf properitäre Software und deren Weiterverkauf eingegangen wurde sollen nun Open Source Lizenzen betrachtet werden. Welche Lizenzen kann ein Softwareentwickler w\"ahlen, wenn er sein Programm der Allgemeinheit zur Verf\"ugung stellen will. \todo{Passt der Satz hier???} In diesem Kapitel wird erl\"autert welche Kriterien f\"ur einen Vergleich der Lizenzen herangezogen werden und warum diese Kriterien Relevanz f\"ur Softwareentwickler besitzen.

Der Softwareentwickler w\"ahlt vor dem Vertrieb seiner Software eine Lizenz aus, unter der er seine Software ver\"offentlichen will. Mit der Lizenz wird geregelt, welche Nutzungsrechte f\"ur die Software einger\"aumt werden. F\"ur die Wahl der Lizenz muss der Entwickler also entscheiden was mit seiner Software getan werden darf. Dies ist vor allem bei Open Source interessant, da die Offenlegung des Quellcodes dem Nutzer mehr M\"oglichkeiten bieten soll. Um diese M\"oglichkeiten zu erlauben, m\"ussen die Rechte in einer Lizenz einger\"aumt werden, da andere sonst gegen die Urheberrechte verstoßen k\"onnten. Was erlaubt wird und was nicht und unter welchen Bedingungen beispielsweise Weiterentwicklungen erm\"oglicht werden, wird durch die Lizenz gesteuert. 

(https://www.gnu.org/philosophy/free-sw) Nach der GNU Philosophie muss freie Software dem Nutzer vier wesentliche Freiheiten gew\"ahren. Wenn diese Freiheiten gew\"ahrt werden, kann die Software als freie Software bezeichnet werden, “die die Freiheit und Gemeinschaft der Nutzer respektiert”. 

(https://www.gnu.org/philosophy/free-sw) 
% (\citealt[][S.\,0]{key0000})
\begin{seList} 
\item Die Freiheit, das Programm auszuf\"uhren wie man m\"ochte, f\"ur jeden Zweck (Freiheit 0).
\item Die Freiheit, die Funktionsweise des Programms zu untersuchen und eigenen Datenverarbeitungbed\"urfnissen anzupassen (Freiheit 1). Der Zugang zum Quellcode ist daf\"ur Voraussetzung.
\item Die Freiheit, das Programm weiterzuverbreiten und damit seinen Mitmenschen zu helfen (Freiheit 2).
\item Die Freiheit, das Programm zu verbessern und diese Verbesserungen der \"Offentlichkeit freizugeben, damit die gesamte Gemeinschaft davon profitiert (Freiheit 3). Der Zugang zum Quellcode ist daf\"ur Voraussetzung.
\end{seList}
Diese Freiheiten werden als Kriterium “freie Software” in den Vergleich eingehen. Nur wenn die vier Freiheiten erf\"ullt sind, kann eine Lizenz als Lizenz f\"ur freie Software gewertet werden. 

(http://www.gnu.org/licenses/copyleft.de.html) Die Offenlegung des Quellcodes erm\"oglicht Weiterentwicklungen und Modifizierungen am Programm. In welcher Form diese weiterverbreitet werden k\"onnen, kann ebenfalls durch die Lizenz geregelt werden. In diesem Zusammenhang ist die Betrachtung des sogenannte Copyleft interessant. Copyleft ist im Prinzip die Umkehrung des Copyrights, welches das Werk vor \"Anderungen sch\"utzt. Mit dem Copyleft wird das Recht den “Quellcode des Programms [...] zu nutzen, zu modifizieren und weiterzuverarbeiten [nur dann einger\"aumt], wenn die Vertriebsbedingungen unver\"andert bleiben.” Es wird also sichergestellt, dass zuk\"unftige Weiterentwicklungen nicht Properit\"ar werden k\"onnen. 

(http://www.heise.de/open/artikel/Open-Source-Lizenzen-221957.html ) Das Copyleft kann zus\"atzlich noch in starkes Copyleft und schwaches Copyleft unterschieden werden. Das starke Copyleft verbietet es properit\"arer Software gegen das Programm / die Bibliothek zu linken. Das heißt es werden sogar Aufrufe durch fremde Software, die nicht unter den gleichen Vertriebsbedingungen stehen verhindert. Das schwache Copyleft erlaubt es - im Gegensatz zum starken Copyleft - dass properit\"arer Code gegen das Programm linken darf. Das bedeutet Schnittstellen des Programms k\"onnen verwendet werden. In beiden F\"allen werden jedoch Modifikationen und Weiterentwicklungen am Programm / der Bibliothek selbst gesch\"utzt, sodass hier die Vertriebsbedingungen gleich bleiben. 

(http://www.oreilly.de/catalog/gplger/chapter/ch02.pdf) Ob das Linken bei starkem Copyleft erlaubt ist oder nicht ist davon abh\"angig, ob das Programm als abgeleitetes Werk bezeichnet werden kann. “Es ist daher notwendig, inhaltlich und funktional zu bewerten, ob zwei Softwarebestandteile eine Einheit bilden oder ob ihnen selbstst\"andige und unabh\"angige Funktionen zukommen. Im Einzelfall kann dies zu schwierigen Auslegungsfragen f\"uhren, die sich teil nicht immer zweifelsfrei beantworten lassen.“

(https://sfconservancy.org/news/2015/oct/28/vmware-update/) Wie die Auslegung des starken Copylefts ausgestaltet wird, kann in einem laufenden Verfahren gegen VMware beobachtet werden. Die Verhandlung ist f\"ur das erste Quartal 2016 angesetzt. Gekl\"art werden muss, ob Software Bestandteile unter GPL gestellt werden m\"ussen oder nicht, wenn sie gegen Teile des Linux Kernels linken. Im konkreten Fall handelt es sich um Kernelmodule die im gleichen Kontext laufen, wie der Copyleft lizenzierte Programmcode.

Die meisten Open Source-Lizenzen enthalten Haftungsausschl\"usse, da Open Source Software unentgeltlich zur Verf\"ugung gestellt wird und die Entwickler deshalb eine Haftung ablehnen. (ISBN 3-89842-606-8 Seite 134f) In Deutschland sind diese Haftungsauschl\"usse jedoch unwirksam, es sei denn der Entwickler bietet die Software vollst\"andig kostenlos an, da in diesem Fall die Regelungen eines Schenkungsvertrags Anwendung finden. (http://blog-it-recht.de/2013/01/24/softwarentwickler-haftung-fur-open-source-software-auf-basis-der-gpl/). Da die Gew\"ahrleistungs- und Haftungsauslussklauseln in Deutschland unwirksam sind, sollen sie nicht untersucht werden.

Im nachfolgenden Vergleich wird also einerseits darauf eingegangen ob es sich um eine freie Lizenz nach der Definition der Free Software Foundation handelt und auf der Seite ob es sich um eine Lizenz mit Copyleft handelt. 