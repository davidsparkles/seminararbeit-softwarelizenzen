Die Vergabe von Lizenzen ist ein Mittel zur Übertragung von Rechten an nichtmateriellen Gütern vom Eigentümer auf einen anderen. \seFootcite{vgl.}{S. 173}{RW:EPL}

Diese Rechte sind durch das Gesetz in unterschiedliche Arten aufgeteilt, von denen das Patentrecht, das Gebrauchsmusterrecht, das Designrecht und das Urheberrecht im folgenden kurz beschrieben wird.
Sowohl Gebrauchsmuster als auch Patente beziehen sich auf Erfindungen die gemäß §1 Abs. 1 PatG und §1 Abs.1 GebrMG neu und gewerblich nutzbar sind,
wobei Patente auf “Erfindungen auf allen Gebieten der Technik” und Gebrauchsmuster auf Erfindungen, die “auf einem erfinderischen Schritt beruhen”, vergeben werden.
Weiterführend ist das Design gemäß §1 des Designgesetzes (kurz DesingG) das Erscheinungbild “eines ganzen Erzeugnisses oder eines Teiles davon” und ist gemäß §2 DesignG geschützt,
wenn es neu ist und Eigenart hat und zusätzlich dazu auch wie Patente und Gebrauchsmuster eingetragen ist. Im Gegensatz dazu werden Urheberrechte nicht in ein Register eingetragen und
beziehen sich gemäß des Urhebergesetzes (kurz UrhG) auf Werke der Literatur, Wissenschaft und Kultur, wozu gemäß §2 Abs. 1 UrhG auch Computerprogramme zählen. \seFootcite{vgl.}{Kapitel 1}{RW:EPL}

Somit kann der Erfinder oder Schöpfer eines schützenswerten immateriellen Gutes gemäß der oben genannten Gesetze über seine Erfindung oder Werk verfügen,
wenn es entsprechenderweise eingetragen oder verbreitet ist. Ist die Erfindung letztendlich geschützt so kann zunächst der Eigentümer des Schutzrechtes ausschließlich über dieses verfügen.
Kann der Erfinder seine Erfindung selbst gewerblich verwerten, dann wäre es für ihn der günstigste Fall. Jedoch ist es häufig so, dass der Erfinder nicht das nötige Wissen und Kapital besitzt,
um seine Erfindung selbst gewerblich verwerten zu können, sodass es sich in solch einem Fall eher für ihn lohnt, das Schutzrecht oder das Nutzungsrecht zu verkaufen. \seFootcite{vgl.}{S. 165 ff.}{RW:EPL}

Zu diesem Zweck müssen Verkäufer und Käufer einen Verwertungsvertrag aushandeln, in dem die Vertragsleistungen, die Gegenleistungen und die Abreden vereinbart werden.
Dabei beschriebt die Vertragsleistung die Übertragung der Verwertungsrechte, die Gegenleistung wird meist in Form einer Zahlung erbracht und die Abrede enthält Informationen unter Anderem
zu Abrechnungsfristen, die Vertragslaufzeit oder auch ein Kündigungsrecht zum Aufheben des Verwertungsvertrages. Bei der Erstellung eines solchen Verwertungsvertrages sollte darauf geachtet werden,
ihn möglichst klar, ausführlich und vor Allem unter Berücksichtigung aller erdenklichen Möglichkeiten zu formulieren, um so späteren Konflikten durch “Vertragslücken” vorzubeugen.
Wird in dem Verwertungsvertrag eine unbeschränkte Rechteübertragung vereinbart, so werden dem Vertragspartner nicht nur die Benutzungsrechte, sondern auch die Verfügungsrechte übertragen.
Das bedeutet, dass der Vertragspartner nach Erhalt der Verfügungsrechte, beispielsweise eines Patentes, dieses Patent weiterverkaufen könnte. Um weiterhin das Eigentum an den Verfügungsrechten
zu behalten, werden lediglich die Benutzungsrechte, auch Lizenzrechte genannt, in einem beschränkten Verwertungsvertrag übertragen. \seFootcite{vgl.}{S. 165 ff.}{RW:EPL}

Lizenzrechte können unter verschiedenen Beschränkungen übertragen werden, sodass Lizenzenrechte häufig zeitlich, örtlich und beispielsweise auch nach der Benutzunsgsart eingeschränkt werden kann.
Neben den eben genannten Einschränkungen werden Lizenzrechte auch in ausschließliche und einfache Lizenzen unterschieden, wobei bei ausschließlichen Lizenzen der Lizenznehmer als einziger meist
auch unter Ausschluss des Lizenzgebers die Benutzungsrechte hält. Im Gegensatz dazu kann der Lizenzgeber bei einer einfachen Lizenz diese weiterhin selbst nutzen und auch weiteren Lizenznehmern
anbieten. Beispielsweise überträgt ein Autor seinem Verlag im Regelfall ein ausschließliches Benutzungsrecht sein Buch zu drucken, um somit auszuschließen, dass das Werk auch von anderen Verlägen
gedruckt wird. Demgegenüber wird eine einfaches Benutzungsrecht von zum Beispiel einen Softwarehersteller auf seine Kunden übertragen, damit gewährleistet ist, dass der Softwarehersteller mehrere
Lizenzen vergeben kann. \seFootcite{vgl.}{S. 173 ff.}{RW:EPL}

Die freie Gestaltung von Lizenzverträgen wird durch Vorschriften gegen Wettbewerbsbeschränkung seitens des Gesetzes gegen Wettbewerbsbeschränkung in Deutschland und auch seitens des EG-Vertrages
eingeschränkt. In dem EG-Vertrag heißt es beispielsweise, dass “alle Vereinbarungen zwischen Unternehmen [...] mit dem Gemeinsamen Markt unvereinbar und verboten sind, [...] welche den Handel
zwischen Mitgliedstaaten zu beeinträchtigen geeignet sind” (Artikel 81 Abs. 1 EG-Vertrag). Somit können Lizenzverträge nicht vollständig frei vereinbart werden. \seFootcite{vgl.}{S. 187 ff.}{RW:EPL}
