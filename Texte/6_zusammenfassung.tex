% Ziel dieser Arbeit war es, die rechtlichen Probleme von Software Lizenzen zu betrachten und Open Source Lizenzen zu vergleichen. 

Die fünf am weitesten verbreiteten Open Source Lizenzen auf Github gew\"ahren jedem Nutzer die vier Freiheiten um als freie Software zu gelten. Somit gilt f\"ur die meiste ver\"offentlichte Software auf dieser Plattform, dass man sie ausf\"uhren, untersuchen, weiterverbreiten und verbessern darf. Bei freier Software kann der Nutzer das Programm kontrollieren, im Gegensatz zur properit\"aren Software, bei der der Eigentümer die Kontrolle \"uber die Funktionsweise hat. 

Möchte man als Entwickler, dass jeder von diesen Freiheiten, auch f\"ur zuk\"unftige Versionen, profitiert, so kann eine Lizenz mit Copyleft gew\"ahlt werden. Im Vergleich kamen hierf\"ur die GPLv2 und die aktuelle GPLv3 in Frage. F\"ur Erstlizensierung mit Copyleft ist die GPLv3 zu bevorzugen, da sie L\"ucken schließt, die den Nutzer trotz Copyleft einschr\"anken k\"onnten. 